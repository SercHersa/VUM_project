\documentclass[12pt,letterpaper]{book}
\usepackage[utf8]{inputenc}
\usepackage[spanish]{babel}
\usepackage{amsmath}
\usepackage{amsfonts}
\usepackage{amssymb}
\usepackage{makeidx}
\usepackage{graphicx}
\usepackage[left=2cm,right=2cm,top=2cm,bottom=2cm]{geometry}

\usepackage{fancyhdr}
\pagestyle{fancy}
\fancyhead{} %Eliminar definiciones previas
\fancyhead[R]{Diseño de un VUM para uso en paquetería: Aplicación de un sistema AGV} 
\fancyfoot[C]{\thepage}


%\includegraphics[scale=•]{•}

\author{Sergio Hernández Sánchez}
\title{Diseño de un Vehiculo Última Milla para uso en paquetería: Aplicación de un sistema AGV}

\date{10 de marzo de 2020}
\begin{document}

\begin{titlepage}
\begin{center}
\includegraphics[scale=1]{Figures/Logo_azul.png}\\
\vspace*{2\baselineskip}
\begin{large}
\textbf{UNIVERSIDAD NACIONAL AUTÓNOMA DE MÉXICO}
\end{large}

PROGRAMA DE MAESTRÍA Y DOCTORADO EN INGENIERÍA

MECÁNICA – DISEÑO MECÁNICO

\vspace*{7\baselineskip}

DISEÑO DE UN VEHÍCULO DE ÚLTIMA MILLA PARA SU USO EN PAQUETERÍA: APLICACIÓN DE UN SUSTEMA AGV

\vspace*{4\baselineskip}

TESIS

QUE PARA OPTAR POR EL GRADO DE:

MAESTRO EN INGENIERÍA

\vspace*{5\baselineskip}

PRESENTA:

SERGIO HERNÁNDEZ SÁNCHEZ

\vspace*{4\baselineskip}

TUTOR PRINCIPAL

DR. ALEJANDRO C. RAMÍREZ REIVICH

\vspace*{4\baselineskip}

CIUDAD UNIVERSITARIA, CDMX, MARZO 2020

\end{center}
\end{titlepage}
%\newpage

\chapter*{AGRADECIMIENTOS}

ASDASDASD
\newpage

\chapter*{RESUMEN}

ASDGSTGHFDTRGSRV
\newpage

\renewcommand{\contentsname}{TABLA DE CONTENIDO}
\tableofcontents
\thispagestyle{empty}
\newpage

\setcounter{page}{1}
\chapter{DEFINICIÓN DEL PROYECTO}
\section{Introducción}
\section{Antecedentes}
Última Milla y retos que enfrenta la paquetería
\section{Métodología}
Ciclos y funciones y Enfoque
\section{Trabajo previo}
Descripción de las tesis anteriores
\section{Planteamiento del problema}
¿Cuál es la problemática?
\section{Objetivo}
General y particulares
\section{Alcances}
A qué se llegará
\section{Equipo de trabajo}
Por etapas
\newpage

\chapter{CICLO 1. USUARIO}
\section{Reto}
\section{Usuario en contexto}
\subsection{Observaciones}
\subsection{Entrevistas}
\subsection{Organizar y jerarquización de necesidades}
\section{Soluciones actuales (mercado)}
\section{Contexto pasado, presente y futuro}
\subsection{Estadísticas, normas, reglamentos}
\section{Factores críticos, hallazgos}
\subsection{Necesidades seleccionadas}

\newpage

\chapter{CICLO 2. EXPERIENCIA}
\section{Redefinición del reto}
\section{Necesidad jerarquizada}
\section{Objetivo (propuesta de valor)}
\section{Requerimientos}
\section{Escenarios}
\section{Personajes}
\section{Mapa de ruta}
\section{Diseño de experiencias}
\section{Nuevas tecnologías}
\section{Factores críticos, hallazgos}

\newpage

\chapter{CICLO 3. PRODUCTO}
\section{Redefinición del reto}
\section{Principios de diseño}
\section{Requerimientos y especificaciones}
\section{Generación de conceptos}
\section{Evaluación de conceptos}
\section{Selección de concepto}

\newpage

\chapter{CICLO 4. PROTOTIPO}
\section{Pruebas con usuarios}
\section{Factores críticos y hallazgos}

\newpage

\chapter{IDEAS DE MI PARTE}
\section{DEFINIR. Reto, problemática, objetivo}

Un problema dificil de solucionar es la cultura deshonesta de algunos trabajadores, ya que terminan provocando problemas que afectan tanto la logistica de la entrega y producen perdidas económicas que afectan a la empresa de reparta y la empresa de venta.

Algunos de los problemas reportados por los usuarios y cliente son los siguientes:

Se roban las cosas

Se roban paquetes

Significan un gasto elevado para las empresas de paquetería

Modifican sus rutas de entregas para ir a lugares de su interés como comer, visitar a su novia, mamá, etc

Siendo seleccionado la experiencia de "madre nodriza", otro problema surgirá, el cómo llegarán los repartidores a ese punto y los problemas que ésto pueda causar ya que vienen de diferentes partes de la ciudad y área metropolitana, por lo que podrían afectar la logistica planificada de entrega.

\section{CONOCER. Contexto, estado del arte, benchmarking, análogos y homólogos}

\section{GENERAR. Brainstorming}

\section{PROBAR. Simuladores, maquetas, prototipos}

\section{APRENDER. Análisis de hallazgos}

\newpage

\chapter{CONCLUSIONES}

\chapter{TRABAJO A FUTURO}




\end{document}