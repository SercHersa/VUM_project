\chapter{IDEAS DE MI PARTE}
\section{DEFINIR. Reto, problemática, objetivo}
\subsection{Reto}

Se plantea que a este vehículo de última milla se le añadan tecnólogías que permitan eficientar el proceso de entrega de paquetería, una primera iteración puede ser la aplicación de un sistema AGV (\textit{Automatic Guided Vehicle}, por sus siglas en inglés). Esto con la idea de tener un primer acercamiento con tecnologías revolucionarias hoy en día, como lo son vehículos autónomos.

\subsection{Problemática}

Retomando algunas de las necesidades obtenidas de nuestros usuarios, nos dimos cuenta que derivan de situaciones culturales del escenario y contexto mismos. Siendo nuestro usuario principal en este ciclo, el cliente, que como se enunció anteriormente es la dueña de la empresa Re!, ya que enució algunas de las problemáticas siguientes:

\begin{itemize}
\item Los empleados no pueden acceder con mochilas porque se pueden robar los paquetes.

\item Hacen mal uso del combustible del vehículo, ya sea porque se desvían de la ruta por intereses personales o porque cargan menos cantidad de éste, mientras que reportan haber comprado una mayor cantidad.

\item En la solución planteada, si dicho VUM tiene una velocidad mayor que una bicicleta, puede ser peligroso para ellos mismos, ya que no tenderán a ir siempre rápido, propiciando accidentes.

\item No es conveniente que el VUM tengra piezas que se quiten, ya que los mismos empleados podrían robárselas.
\end{itemize}
Con estos hallazgos, se hace evidente que el vehículo y el proceso de entrega es altamente mejorable.

\subsection{Objetivo}

Mejorar la Eficiencia del sistema de entrega VUM, planteando la implementación de un sistema de vehículo de Guiado Automático (AGV) en rutas controladas.

\subsection{Sketch}

///Un problema dificil de solucionar es la cultura deshonesta de algunos trabajadores, ya que terminan provocando problemas que afectan tanto la logistica de la entrega y producen perdidas económicas que afectan a la empresa de reparta y la empresa de venta.

Algunos de los problemas reportados por los usuarios y cliente son los siguientes:

Se roban las cosas

Se roban paquetes

Significan un gasto elevado para las empresas de paquetería

Modifican sus rutas de entregas para ir a lugares de su interés como comer, visitar a su novia, mamá, etc.

Ya que el vehículo tiene la capacidad de ir a más de 25 km/hr, puede ser una velocidad a la que sea peligroso para el mismo repartidor, por lo que se necesitaría un sistema de conducción de asistencia que procure evitar accidentes.

Siendo seleccionado la experiencia de "madre nodriza", otro problema surgirá, el cómo llegarán los repartidores a ese punto y los problemas que ésto pueda causar ya que vienen de diferentes partes de la ciudad y área metropolitana, por lo que podrían afectar la logistica planificada de entrega.

\textbf{Objetivo}

El objetivo será, hacer que un sistema de vehículo automaticamente guiado, (AGV por sus siglas en inglés), sea implementado en el VUMI, para esto se necesitan varias cosas:

\textbf{Accionamiento}

Controlar velocidad de movimiento

Controlar la dirección

Controlar el frenado

\textbf{Sensado}

Conocer la velocidad del vehículo

Conocer la posición angular del volante

Conocer la inclinación del vehiculo para saber si no se ha caido

Conocer si hay un obstaculo frente a él para saber si se debe frenar

Seguir una trayectoria, ya sea por una línea negra en el piso y esta sea detectada por un sensor fotoeléctrico, por una linea magnética y sensar la variación magnética o alguna otra forma de seguimiento.

\textbf{Procesamiento}

Esto se realizará dados los datos sensados y se enviarán señales a los actuadores para ealizar una rutina definida.

Solo será un sistema reactivo, es decir, reaccionará a ciertos eventos previstos, pero no tendrá sistema de inteligencia que genere que pueda resolver una problemática en particular.




\section{CONOCER. Contexto, estado del arte, benchmarking, análogos y homólogos}

\section{GENERAR. Brainstorming}

\section{PROBAR. Simuladores, maquetas, prototipos}

\section{APRENDER. Análisis de hallazgos}
