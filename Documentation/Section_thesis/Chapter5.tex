\chapter{CICLO 4. PROTOTIPO}

\section{Redefinición del reto}

Se plantea que al VUM se le añadan tecnologías que permitan eficientar el proceso de entrega de paquetería, una primera iteración puede ser la aplicación de un sistema AGV (\textit{Automatic Guided Vehicle}, por sus siglas en inglés). Esto con la idea de tener un primer acercamiento con tecnologías revolucionarias hoy en día, como lo son los vehículos autónomos.

\section{Problemática}

Retomando algunas de las necesidades obtenidas de nuestros usuarios, nos dimos cuenta que derivan de situaciones culturales del escenario y contexto mismos. Siendo nuestro usuario principal en este ciclo, el cliente, que como se enunció anteriormente es la dueña de la empresa Re!, ya que enució algunas de las problemáticas siguientes:

\begin{itemize}
\item Los empleados no pueden acceder con mochilas porque se pueden robar los paquetes.

\item Hacen mal uso del combustible del vehículo, ya sea porque se desvían de la ruta por intereses personales o porque cargan menos cantidad de éste, mientras que reportan haber comprado una mayor cantidad.

\item En la solución planteada, si dicho VUM tiene una velocidad mayor que una bicicleta, puede ser peligroso para ellos mismos, ya que no tenderán a ir siempre rápido, propiciando accidentes.

\item No es conveniente que el VUM tenga piezas que se quiten, ya que los mismos empleados podrían robárselas.

\item Siendo seleccionado la experiencia de "madre nodriza", otro problema surgirá, el cómo llegarán los repartidores a ese punto y los problemas que ésto pueda causar ya que vienen de diferentes partes de la ciudad y área metropolitana, por lo que podrían afectar la logistica planificada de entrega.

\end{itemize}
Con estos hallazgos, se hace evidente que el vehículo y el proceso de entrega es altamente mejorable.

\section{Objetivo}

Mejorar la eficiencia del sistema de entrega VUM, planteando la implementación de un sistema de Vehículo de Guiado Automático (AGV) en rutas controladas.

\section{CONOCER. Contexto, estado del arte, benchmarking, análogos y homólogos}

\subsection{Por definir}
\begin{itemize}
\item Sistema para que no se caiga
\item Actuación

\begin{itemize}
\item Aceleración
\item Dirección
\item Frenado
\item Luces
\end{itemize}

Sensado
\begin{itemize}
\item Detección de obstáculos
\item Detección de ruta
\item Conocer el estado del VEnTUM
\end{itemize}

Procesamiento (Programación para las diferentes condiciones de funcionamiento)

\end{itemize}




\section{GENERAR. Generación de conceptos}
Generación de conceptos y selección
\section{GENERAR. Selección de conceptos}
\section{GENERAR. Desarrollo de la propuesta}
\section{PROBAR. Simuladores, maquetas, prototipos}

\section{APRENDER. Análisis de hallazgos}
